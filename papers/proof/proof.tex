\documentclass[a4paper, 10pt]{article}
\usepackage{amssymb}
\usepackage{amsthm}
\usepackage{amsmath}
\usepackage{natbib}

\title{Modelling Cluedo: Validity of private announcements}
\author{Laura van de Braak \and Luuk Boulogne \and Ren\'e Mellema}
\date{}

\newcommand{\impl}{\rightarrow}
\newcommand{\A}{\mathbf{A}}
\newcommand{\DBox}[1]{\left[#1\right]}
\newcommand{\DDiamond}[1]{\left\langle #1 \right\rangle}

\newtheorem{prop}{Proposition}

\begin{document}

\maketitle

For our implementation of Cluedo, we needed to implement private
announcements \citep[see][Chapter 6]{ditmarsch2007dynamic}. We decided to
implement this by removing the relations for the agent that received the
announcement between the states where the announcement holds and the states
where the announcement does not hold. So if agent $i$ receives announcement
$\varphi$ and $R_i$ is the set of relations before the announcement then
$R_i'$, the set of relations for agent $i$ after the private announcement.
$R_i'$ is defined as $R_i = \{(s, t) | (s, t) \in R_i, (M', s) \models
\varphi \text{ if and only if } (M', t) \models \varphi\}$.

We denote propositional variables as $p_i^j$, where $p$ stands for the
category of the card, $i$ for the number of that card in the category and
$j$ for the agent that has that card. So $p_3^0$ means that person card
number 3 is dealt to agent 0. In this notation, agent 0 is the envelope.
If the number or category of the card is not important, the number is left
out and the letter is not $p$ or $w$.  We also have a model $M$, which
after the first public announcement becomes $M'$, after the private
announcement becomes $M''$, and after the second public announcement becomes
$M'''$.

Most of the rules for Cluedo have been implemented in the model implicitly,
such as the rule $c^i \impl \neg c^j$ with $i \not = j$, which says that only
one person can have a specific card. Also the rule that everybody has two
cards is implemented in this way.

Assume that we have a set of agents $\A$. Now agent $i \in \A$ speaks an
accusation, which is a set of cards denoted as $a$. Now the agent $j \in
\A, i \not = j$ says he has one of these cards, $c \in a$, and shows that
card $c$ to $i$.  In our implementation, this is modelled as the public
announcement $\bigvee_{k \in a} k^j$. After this public announcement, the
private announcement $c^j$ is made towards agent $i$. After that, the
public announcement $\bigvee_{k \in a} E_{i, j} k^j$ is made, to show that
both agents know the same card. 

During the private announcement, only agent $i$ should get new knowledge,
while we would also want that during the public announcements it becomes
common knowledge that agent $j$ has at least one of the cards in $a$ and
that agent $i$ and agent $j$ now know the same card. This idea is captured
in the five propositions.

\begin{prop}
    $(M''', s0) \models C(\bigvee_{k \in a} \neg k^0)$
\end{prop}
\begin{prop}
    $(M''', s0) \models C_{i, j} \neg c^0$
\end{prop}
\begin{prop}
    $(M''', s0) \models C(\bigvee_{k \in a} C_{i,j} \neg k^0)$
\end{prop}
\begin{prop}
    For all agents $h \in \A$ with $h \not = i$ and $h \not = j$ it holds
    that for all cards $k \in a$: \[ (M', s0) \models M_h k^0 \text{
            implies } (M'', s0) \models M_h k^0 \]
\end{prop}
\begin{prop}
    For all agents $h \in \A$ with $h \not = i$ and $h \not = j$ it holds
    that for all cards $k \in a$: \[ (M', s0) \models \neg K_h k^j \text{
            implies } (M'', s0) \models \neg K_h k^j \]
\end{prop}

\begin{proof}[Proof of Proposition 1]
    First we will prove that $(M', s0) \models \DBox{\bigvee_{k \in a} k^j}
    C(\bigvee_{k \in a} k^j)$. For this, we first have to see that this
    announcement is part of the language $\mathcal{L}_N^{u0}$ from
    \cite{ditmarsch2006secret}. This means that we get $\DBox{\bigvee_{k
            \in a} k^j} \bigvee_{k \in a} k^j$. From propositional logic we
    now have $\top \impl \DBox{\bigvee_{k \in a} k^j} \bigvee_{k \in a}
    k^j$. Since agent $j$ can only respond to an accusation if it has one
    of the cards in the accusation, we know that $c^j$ for some $c \in a$.
    So $\bigvee_{k \in a} k^j$. This means we also get $(\top \land
    \bigvee_{k \in a} k^j) \impl E\top$. Now we can apply Proposition 4.26
    from \citet{ditmarsch2007dynamic} to get $\top \impl \DBox{\bigvee_{k
            \in a} k^j} C\left(\bigvee_{k \in a} k^j\right)$. Now by a
    propositional tautology we get $\DBox{\bigvee_{k \in a} k^j}
    C\left(\bigvee_{k \in a} k^j\right)$.

    Since $M'$ is the model after the first public announcement and we have
    $(M, s0) \models \DBox{\bigvee_{k \in a} k^j} C\left(\bigvee_{k \in a}
        k^j\right)$. This means that we also have $(M', s0) \models
    C\left(\bigvee_{k \in a} k^j\right)$.  Since we also have $k^j \impl
    \neg k^0$, via propositional logic we have $\left(\bigvee_{k \in a} k^j
    \right) \impl \left(\bigvee_{k \in a} \neg k^0\right)$, and if we apply
    (R3) we get $C\left(\left(\bigvee_{k \in a} k^j \right) \impl
        \left(\bigvee_{k \in a} \neg k^0\right)\right)$. Now we can apply
    modus ponens in order to get $C\left(\bigvee_{k \in a} \neg
        k^0\right)$. Since the logic \textit{PAC} is sound and complete
    with respect to \textbf{PAC}, we can conclude that
    $(M', s0) \models C\left(\bigvee_{k \in a} \neg k^0\right)$.
\end{proof}

\begin{proof}[Proof of Proposition 2]
    After we made the private announcement $\langle i, c^j\rangle$, we are
    in model $M''$. In this model, there are no relations between states
    $s, t \in S$ if $(M', s) \models c^j$ and $(M', t) \not \models c^j$
    for the agents $i$ and $j$. This means that there is also no $t$ such
    that $s0 \twoheadrightarrow_{i,j} t$ with $(M'', t) \models \neg c^j$.
    Therefore, for all worlds $s0 \twoheadrightarrow_{i,j} u$ we have
    $(M'', u) \models c^j$. By the definition of $C$ we get $(M'', s0)
    \models C_{i,j} c^j$. Since we also have $c^j \impl \neg c^0$, which
    holds in each state, so $(M'', s0) \models C_{i,j} (c^j \impl \neg
    c^0)$. Now by modus ponens we get $(M'', s0) \models C_{i,j} \neg c^0$.
\end{proof}

\begin{proof}[Proof of Proposition 3]
    Here we make the observation that the second public announcement
    $\bigvee_{k\in a} E_{i,j} k^j$ is also a member of
    $\mathcal{L}_N^{u0}$ and that the proof will be similar to the proof of
    Proposition 1.

    After we have gotten $(M''', s0) \models C\left(\bigvee_{k \in a }
        E_{i,j} k^j\right)$, we now need to promote the $E$ operator to the $C$
    operator. For this, we can use the rule $(\varphi \land E \varphi)
    \impl C \varphi$ and the rule $E \varphi \impl \varphi$. Now by
    propositional logic we get $\bigwedge_{k \in a} \left(E_{i,j}k^j \impl
        k^j\right)$. Using that and the initial formula we get $(M''', s0)
    \models C\left(\bigvee_{k \in a } E_{i,j} k^j \land k^j\right)$. Now we
    can conclude $(M''', s0) \models C\left(\bigvee_{k \in a} C_{i,j}
        k^j\right)$.
\end{proof}

\begin{proof}[Proof of Proposition 4]
    We will prove this by contraposition.
    Take an agent $h \in \A$ such that $h \not = i$ and $h \not = j$ and
    $(M', s0) \models M_h k^0$ for some $k \in a$. Now assume that $(M'',
    s0) \not \models M_h k^0$. This means that during the private
    announcement for $i$, the relations for $h$ have changed. By
    the definition of private announcements, this is not what happens, so
    that means that our assumption did not hold. So $(M', s0) \models M_h
    k^0$ implies $(M'', s0) \models M_h k^0$ for some $k \in a$. Since $k$
    was also chosen arbitrarily, this holds for all cards in $a$. 
\end{proof}

\begin{proof}[Proof of Proposition 5]
    We will prove this by contraposition.
    Take an agent $h \in \A$ such that $h \not = i$ and $h \not = j$ and
    $(M', s0) \models \neg K_h k^j$ for some $k \in a$. Now assume that
    $(M'', s0) \not \models \neg K_h k^j$. This means that during the private
    announcement for $i$, the relations for $h$ have changed. By
    the definition of private announcements, this is not what happens, so
    that means that our assumption did not hold. So $(M', s0) \models \neg
    K_h k^j$ implies $(M'', s0) \models \neg K_h k^j$ for some $k \in a$.
    Since $k$ was also chosen arbitrarily, this holds for all cards in $a$. 
\end{proof}

\bibliographystyle{plainnat}
\bibliography{../references}

\end{document}
