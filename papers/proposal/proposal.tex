\documentclass[a4paper, 10pt]{article}
%\usepackage{fullpage}
\usepackage{natbib}

\title{Modelling Cluedo}
\author{Laura van de Braak \and Luuk Boulogne \and Ren\'e Mellema}
\date{}

\newcommand{\term}{[INSERT TERM HERE]}

\newcommand{\abov}[2]{\left(\begin{array}{c} #1 \\ #2 \end{array}\right)}

\begin{document}
\maketitle

For this project, we want to implement a dynamic epistemic modelling tool
for a simplified version of Cluedo. This tool will be able to check
formula's at a specific Kripke model, it will be able to generate these
models from a card deal, and it will be able to update with given
information. This tool can then be used to analyse specific situations in
Cluedo.

\section{Simplification}
Because the full scope of cluedo would be to complex to check formulas in a
reasonable amount of time, we decided to simplify our version of Cluedo to
a game with only 6 weapons, 4 persons, and no rooms. This also means that
there are only 4 agents (players) who have 2 cards each. There are also
only 2 cards in the envelope. Since we have no rooms in the game, the
players also do not have to move around. The other rules for the game stay
the same \citep{cluedo}.

This drastically reduces the number of states that the game can have, for a
given envelope content, player one gets 2 of the 8 remaining cards, player
2 gets 2 of the 6 then remaining cards etc.~ and therefore the total number
of possible deals for a given envelope content is:
\begin{equation}
    \abov{8}{2} \times \abov{6}{2} \times \abov{4}{2} \times \abov{2}{2} =
    2520
\end{equation}
Before we deal the cards for the players, one weapon and one person is draw
for the envelope. This means that there are $6 \times 4 = 24$ possible
envelope contents, this results in a total number of $60.480$ possible
states.

\section{Possibilities in a turn}
In a turn one player gives a suspicion. In turn the other players say
whether they have one of the cards in the suspicion. If a player does not
have a card in the suspicion, this is a public announcement that they have
neither of these cards. If a player does have one of the cards in the suspicion
they have to reply by saying that they have on of the cards and showing
that card to the player that made the suspicion. This is a message from one
player to the other.

% We gaan cluedo doen

% Door vermindering: 2520 * 24 states

\bibliographystyle{plainnat}
\bibliography{../references}

\end{document}
