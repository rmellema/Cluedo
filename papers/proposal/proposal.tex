\documentclass[a4paper, 10pt]{article}
\usepackage{natbib}

\setcounter{secnumdepth}{0} 

\usepackage{geometry}
%\usepackage{layout}

% \setlength{\voffset}{-0.1in}
\setlength{\headsep}{10pt}



\title{Modelling Cluedo}
\author{Laura van de Braak \and Luuk Boulogne \and Ren\'e Mellema}
\date{}

\newcommand{\term}{[INSERT TERM HERE]}

\newcommand{\abov}[2]{\left(\begin{array}{c} #1 \\ #2 \end{array}\right)}

\begin{document}

\maketitle

\noindent For this project, we want to implement a dynamic epistemic modelling tool
for a simplified version of Cluedo. This tool will be able to check
formulas in a specific Kripke model: it will be able to generate these
models from a dealt hand, and it will be able to update as extra information is given. This tool can then be used to analyse specific situations in
Cluedo.

\subsection{Simplification}
Because implementing the full scope of cluedo would be too complex to check formulas in a reasonable amount of time, we decided to simplify our version of Cluedo to a game with only six weapons, four persons, and no rooms. This means that there are only four agents (players) who have two cards each. Because we use two categories, there are only two cards in the envelope, instead of the usual three. Since we are only interested in the reasoning within this game and movement has little influence on this, we will leave it out of our implementation. The other rules for the game stay the same \citep{cluedo}.

This drastically reduces the number of states the game can have. For a
given envelope content, player 1 gets two of the eight remaining cards, player
2 gets two of the six then remaining cards etc. Therefore the total number
of possible hands dealt for a given envelope content is:
\begin{equation}
    \abov{8}{2} \times \abov{6}{2} \times \abov{4}{2} \times \abov{2}{2} =
    2520
\end{equation}
Before we deal the cards for the players, one weapon and one person is drawn
for the envelope. This means that there are $6 \times 4 = 24$ possible
envelope contents, this results in a total number of $60.480$ possible
states.

\subsection{Possibilities in a turn}
In a turn a player makes a suggestion. In clockwise order the other players say
whether they have one of the cards in the suggestion. If a player does not
have a card in the suggestion, this is a public announcement that they have
neither of these cards. If a player does have one of the cards in the suggestion
they have to reply by saying that they have one of the cards and showing
that card to the player that made the suggestion. This is a message from one
player to the other.

% We gaan cluedo doen

% Door vermindering: 2520 * 24 states

\bibliographystyle{plainnat}
\bibliography{../references}

\end{document}
